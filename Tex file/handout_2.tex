% This sample document was created by P.J. Healy (healy.52@osu.edu) for educational purposes. You may use this as a template for your own documents as you wish.

% Lines that start with a percent sign are just comments - they won't be processed or show up in the final output. If you actually want a percent sign to show up, use \% instead, like ``I got 100\%!''

% \documentclass says what type of document we're making and gives some basic options (options appear in square brackets. Here, our document is like an article (as opposed to a book, report, PhD thesis, etc.), we want 11-point font, and equation numbers on the left (leqno).
\documentclass[11pt,leqno]{article}

% These are the following packages we want LaTeX to load, since we might use them. I group them by similarity.
\usepackage{amsfonts,amsmath,amssymb,amsthm}
\usepackage{color,graphicx}
\usepackage{fullpage,setspace}
\usepackage[colorlinks=true,urlcolor=darkgray,bookmarks=false]{hyperref}

% First we tell LaTeX that we want to create a numbered "theorem-like" environment whose label is "Theorem X" and whose numbering will be handled by an internal counter named "theorem".
\newtheorem{theorem}{Theorem}
\newtheorem{example}{Example}
\newtheorem{definition}{Definition}
\newtheorem{proposition}{Proposition}
\newtheorem{remark}{Remark}

% The 'length variable' \parindent says how far to indent each paragraph. I want to set that length variable to zero:
\setlength{\parindent}{0mm}

% At OSU, you might use scarlet and gray colors for things, so here they are. Note: this requires the color package to be loaded. If you don't use this (and I don't below), you might as well comment it out or delete it.
\definecolor{scarlet}{cmyk}{0,1.00,0.65,0.15}
\definecolor{gray}{cmyk}{0.06,0,0,0.34}

\begin{document}


% First, move up from the normal starting point on the page by 20 millimeters so the table appears in the top margins
\vspace*{-20mm}

% Here's the actual table. We start with a `tabular' environment with 3 columns. The column alignments are left, center, and right, respectively, so the command option is {lcr}. Within the table, columns are delimited by & and rows by \\
% Note that LaTeX ignore consecutive spaces, including tabs, so you can use tabs to make your table look reasonable here.
\begin{tabular*}{\textwidth}{@{\extracolsep{\fill}}lcr}
Econ 8714     & \hfill    &         Professor: P.J. Healy          \\
Microeconomic Theory 2B  &           &   TA: Han Wang    
\end{tabular*}

% Finally, let's put a big `title' in the center of the page, after we `skip' down a bit of space.
\bigskip
\begin{center}
{\Large 03/10/2023 Recitation \#2 Handout}
\end{center}

% Skip some more space and then start working
\bigskip


\textbf{1. Public Goods}

\textbf{1.0. Setting:}
\begin{itemize}
    \item The set of agents $N=\{1,2,\ldots,n\}$.
    \item Public good consumption $y$
    \item Private good consumption $\textbf{x}=(x_{1},\ldots,x_{n})$
    \item Agent $i$'s utility $u_{i}(x_{i},y)=v_{i}(y)+x$ (assume $v_{i}^{\prime\prime}\leq 0$)
    \item The cost of producing public good $c(y)$   (assume $c^{\prime}>0$, $c^{\prime\prime}\geq 0$)
\end{itemize}

\textbf{1.1. Pareto optimal level of public good}

Pareto optimality (+ interior solution) $\implies$ Samuelson condition $\sum_{i} v_{i}^{\prime}(y)=c^{\prime}(y)$.

\textbf{1.2. Private provision of public good}

Consider case where public good provided by means of private purchases by consumers. 

A Walrasian equilibrium is a price vector $(p^{*},1)$\footnote{WLOG, we can normalize the price for the private good to be $1$.} and an allocation $(y^{*},\textbf{x}^{*})$:
\begin{enumerate}
    \item Consumers optimize: 
    
    $y_i^{*} \in \underset{y_i \geq 0}{\operatorname{argmax}} ~v_i\left(y_i+\sum_{j \neq i} y_j\right)-p^* y_i+\theta_{i} \Pi_{0}(y_{0}^{*},p^{*})$ $\implies$ $v_{i}^{\prime}\left(y^* _i+\sum_{j \neq i} y^* _j\right)\leq p^{*}$
    \item Firm optimizes: 
    
    $y_0^{*} \in \underset{y_0 \geq 0}{\operatorname{argmax}} ~\Pi_{0}(y_{0},p^{*}):=p^{*} y_0 -c(y_0)$ $\implies$ $p^{*}\leq c^{\prime}(y_{0}^{*})$
    \item Market clears: $y_0^{*}=\sum_{i} y_i^{*}$ (and $x_{i}^{*}=\omega_{i}-p^* y^{*}_i+\theta_{i} \Pi_{0}(y_{0}^{*},p^{*})$, $\forall i$)
\end{enumerate}

To sum up, $v_{i}^{\prime}\left(y^*\right)\leq p^{*}$, $\forall i$. Draw a graph to illustrate the free-rider issue.

\textbf{1.3. Internalizing the externality}

Consider case where every consumer reports the total amount of the public good she will consume.

A Lindahl equilibrium is a price vector $(\textbf{p}^{*},1)$ such that $\textbf{p}^{*}=(p^* _1,p^* _2,\ldots,p^* _n)$ and an allocation $(y^{*},\textbf{x}^{*})$:

\begin{enumerate}
    \item Consumers optimize: 
    
    $y_i^{*} \in \underset{y_i \geq 0}{\operatorname{argmax}} ~v_i\left(y_i\right)-p_{i}^* y_i+\theta_{i} \Pi_{0}(y_{0}^{*},\textbf{p}^{*})$ $\implies$ $v_{i}^{\prime}\left(y^* _i\right) \leq p_i ^{*}$
    \item Firm optimizes: 
    
    $y_0^{*} \in \underset{y_0 \geq 0}{\operatorname{argmax}} ~\Pi_{0}(y_{0},\textbf{p}^{*}):=\sum_{i} p_{i}^{*} y_0 -c(y_0)$ $\implies$ $\sum_{i} p_{i}^{*} \leq c^{\prime}(y_{0}^{*})$
    \item Market clears: $y_0 ^{*}=y_1 ^{*}=\cdots=y_n ^{*}$ (and $x_{i}^{*}=\omega_{i}-p_{i}^* y^{*}_i+\theta_{i} \Pi_{0}(y_{0}^{*},\textbf{p}^{*})$, $\forall i$)
\end{enumerate}
To sum up, with interior solution, $\sum_{i} v_{i}^{\prime}\left(y^*\right)= c^{\prime}(y^{*})$. [Nice outcome: P.O. \& IR] Draw a graph.

\newpage
\textbf{2. Social Choice}

\textbf{2.0. Definitions}

\begin{example}
Two teaching assistants decide the color of printing papers for the midterm exam.
\end{example}

\begin{itemize}
    \item SWF vs SCF/SCC
    \item Axioms on SWF: Paretian, IIA
    \item Axioms on SCF: Weak Paretian, monotonicity
    \item dictatorial
\end{itemize}

\textbf{2.1. Arrow's impossibility theorem}

We've proved Muller-Satterthwaite theorem (in terms of SCF) during the lecture. Refer to Reny (2000) \textit{Arrow’s Theorem and the Gibbard-Satterthwaite Theorem: A Unified Approach}.

Key assumptions:

URIP3: U(niversal domain) - R(ational) - I(IA) - P(aretian) - 3 (alternatives) 

\begin{itemize}
    \item U: think about majority rule under single-peaked preferences
    \item R: think about Condorcet cycles
    \item I: think about Borda count
    \item P: think about constant SCF
    \item 3: think about majority rule with $|X|=2$
\end{itemize}

\textbf{3. Exercises}
\begin{enumerate}
    \item (HW) Consider the problem of public goods. Prove that at the Walrasian equilibrium, the only contributor is the agent with highest $v_{j}^{\prime}(y^{*})$.
    \item (HW) Check that in the Kolm triangle, if $p_{1}+p_{2}=1$, then price hyperplanes coincide.
    \item (HW) Consider the domain of single-peaked preferences and let $F$ be the SWF under pairwise majority voting. Is $F$ transitive? (What happens if n is odd/even?) Could $F$ have cycles ``below the top''?
    \item (HW) How to construct from $F$ (SWF) to $f$ (SCF) and to $F^{\prime}$ (SWF)?
    Is $F$ the same as $F^{\prime}$?
\end{enumerate}


\end{document} 