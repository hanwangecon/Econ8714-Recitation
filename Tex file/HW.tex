% This sample document was created by P.J. Healy (healy.52@osu.edu) for educational purposes. You may use this as a template for your own documents as you wish.

% Lines that start with a percent sign are just comments - they won't be processed or show up in the final output. If you actually want a percent sign to show up, use \% instead, like ``I got 100\%!''

% \documentclass says what type of document we're making and gives some basic options (options appear in square brackets. Here, our document is like an article (as opposed to a book, report, PhD thesis, etc.), we want 11-point font, and equation numbers on the left (leqno).
\documentclass[11pt,leqno]{article}

% These are the following packages we want LaTeX to load, since we might use them. I group them by similarity.
\usepackage{amsfonts,amsmath,amssymb,amsthm}
\usepackage{color,graphicx}
\usepackage{fullpage,setspace}
\usepackage[colorlinks=true,urlcolor=darkgray,bookmarks=false]{hyperref}

% First we tell LaTeX that we want to create a numbered "theorem-like" environment whose label is "Theorem X" and whose numbering will be handled by an internal counter named "theorem".
\newtheorem{theorem}{Theorem}
\newtheorem{example}{Example}
\newtheorem{definition}{Definition}
\newtheorem{proposition}{Proposition}
\newtheorem{HW}{Question}

% The 'length variable' \parindent says how far to indent each paragraph. I want to set that length variable to zero:
\setlength{\parindent}{0mm}

% At OSU, you might use scarlet and gray colors for things, so here they are. Note: this requires the color package to be loaded. If you don't use this (and I don't below), you might as well comment it out or delete it.
\definecolor{scarlet}{cmyk}{0,1.00,0.65,0.15}
\definecolor{gray}{cmyk}{0.06,0,0,0.34}

\begin{document}


% First, move up from the normal starting point on the page by 20 millimeters so the table appears in the top margins
\vspace*{-20mm}

% Here's the actual table. We start with a `tabular' environment with 3 columns. The column alignments are left, center, and right, respectively, so the command option is {lcr}. Within the table, columns are delimited by & and rows by \\
% Note that LaTeX ignore consecutive spaces, including tabs, so you can use tabs to make your table look reasonable here.
\begin{tabular*}{\textwidth}{@{\extracolsep{\fill}}lcr}
Econ 8714     & \hfill    &         Professor: P.J. Healy          \\
Microeconomic Theory 2B  &           &   TA: Han Wang    
\end{tabular*}

% Finally, let's put a big `title' in the center of the page, after we `skip' down a bit of space.
\bigskip
\begin{center}
{\Large HW}
\end{center}

% Skip some more space and then start working
\bigskip


\textbf{1. Public Goods \& Externalities}

\begin{HW}
Consider the problem of public goods. Prove that at the Walrasian equilibrium, the only contributor is the agent with highest $v_{j}^{\prime}(y^{*})$.
\end{HW}

\begin{HW}
Show that Lindahl equilibrium is IR.
\end{HW}

\begin{HW}
Check that in the Kolm triangle, if $p_{1}+p_{2}=1$, then price hyperplanes coincide.
\end{HW}

\textbf{2. Social Choice}

\begin{HW}
Consider the domain of single-peaked preferences and let $F$ be the SWF under pairwise majority voting. Is $F$ transitive? (What happens if n is odd/even?) Could $F$ have cycles ``below the top''?
\end{HW}

\begin{HW}
We've shown the proof of Muller-Satterthwaite theorem using social choice function. Go through the similar proof but with social welfare function, Paretian and IIA. (Read Reny's paper)
\end{HW}

\begin{HW} Prove or disprove
\begin{enumerate}
    \item $F$ is IIA + PAR $\implies$ $f$ MONO +WP
    \item $F$ is dictatorial $\implies$ $f$ dictatorial
    \item $f$ MONO +WP $\implies$ $F$ is IIA + PAR
    \item $f$ dictatorial $\implies$ $F$ is dictatorial
\end{enumerate}
\end{HW}

\begin{HW}
    How to construct from $F$ (SWF) to $f$ (SCF) and to $F^{\prime}$ (SWF)?
    Is $F$ the same as $F^{\prime}$?
\end{HW}

\textbf{3. Mechanism Design}

\begin{HW}
We've shown that the revelation principle holds for single agent utility maximization. $$F(\theta) \supseteq g\left(\mu_{\Gamma}(\theta)\right) \supseteq g\left(\hat{\mu}_{\Gamma}(\theta)\right)=h(\theta) \in h\left(\mu_{\hat{\Gamma}}(\theta)\right)$$ \begin{enumerate}
    \item Is $h(\theta)=h(\mu_{\hat{\gamma}}(\theta))$ for every $\theta$?
    \item If  $\Gamma=(S, g)$ weakly implements $F$, then $\hat{\Gamma}=\left(\Theta, h \equiv g \circ \hat{\mu}_{\Gamma}\right)$ weakly implements $F$.
\end{enumerate}
\end{HW}

\begin{HW}
    Suppose $f$ is single-valued. If there exists $\Gamma$ that weakly implements $f$, then $f$ is IC. The contra-positive to this result is: If $f$ is not IC, then there is no $\Gamma$ that weakly implements $f$. Could there exist a $\Gamma$ that partially implements f?
\end{HW}

\begin{HW}
    \begin{theorem}
Suppose $SC_{+}$ and uniformly bounded MRS.

$f(\theta)=\left(x(\theta),t(\theta)\right)$ is IC, if and only if 
\begin{enumerate}
    \item $\frac{d x}{d \theta}\geq 0$, $\forall \theta$\\
    \item $\frac{d t}{d \theta}=-\left(\frac{d u/d x}{d u/d t}\right)\cdot\frac{d x}{d \theta}$, $\forall \theta$
\end{enumerate}
\end{theorem}

Prove the ``if'' part of this theorem by contradiction.
\end{HW}

\begin{HW}
    Consider a quasi-linear setting:
\begin{itemize}
\item Principal: $u_{0}(x,t)=v_{0}(x)-t$
\item Agent: $u_{1}(x,t|\theta)=v_{1}(x|\theta)+t$
\item MRS: $\frac{\partial v_{1}}{\partial x}>0$; $SC_{+}$: $\frac{\partial^{2} v_{1}}{\partial x \partial \theta}>0$
\end{itemize}

%\newpage

\begin{align}
    \max_{x(\cdot),t(\cdot)} \quad v_{0}(x(\theta))-t(\theta)\quad \text{s.t. [IR] } v_{1}(x(\theta)|\theta)+t(\theta)\geq 0,~ \forall \theta \tag{First-best}
\end{align}
\begin{itemize}
    \item IR will be binding at every $\theta$.
    \item $x\in \arg \max_{x(\cdot)} \left\{ v_{0}(x(\theta))+ v_{1}(x(\theta)|\theta)\right\}$, so $x$ is P.O.
    \item $x$ is not IC (agent has incentive to lie downwards):
\end{itemize}

\begin{align*}
    &u_{1}(x(\hat{\theta}),t(\hat{\theta})|\theta)=v_{1}(x(\hat{\theta})|\theta)\underbrace{-v_{1}(x(\hat{\theta})|\hat{\theta})}_{=t(\hat{\theta}) ~\text{(binding IR)}}\\
    [\hat{\theta}] \quad &\frac{\partial v_{1}(x(\hat{\theta})|\theta)}{\partial x}\frac{d x(\hat{\theta})}{d \hat{\theta}}-\frac{\partial v_{1}(x(\hat{\theta})|\hat{\theta})}{\partial x}\frac{d x(\hat{\theta})}{d \hat{\theta}}-\frac{\partial v_{1}(x(\hat{\theta})|\hat{\theta})}{\partial \theta}=0\\
    \iff \quad &\underbrace{\left[\frac{\partial v_{1}(x(\hat{\theta})|\theta)}{\partial x}-\frac{\partial v_{1}(x(\hat{\theta})|\hat{\theta})}{\partial x}\right]}_{\geq 0}\frac{d x(\hat{\theta})}{d \hat{\theta}}=\frac{\partial v_{1}(x(\hat{\theta})|\hat{\theta})}{\partial \theta}
    \implies \quad \text{By } SC_{+}, ~\theta \geq \hat{\theta}
\end{align*}
Here, we need to show that the first best solution satisfies $d x(\hat{\theta})/d \hat{\theta}\geq 0$.
\begin{align}
    \max_{x(\cdot),t(\cdot)} \quad \int_{\underset{\bar{}}{\theta}}^{\hat{\theta}} \left[v_{0}(x(\theta))-t(\theta)\right] g(\theta) d\theta \quad \text{s.t. [IR] } &v_{1}(x(\theta)|\theta)+t(\theta)\geq 0,~ \forall \theta \tag{Second-best} \\ \text{[IC] }&v_{1}(x(\theta)|\theta)+t(\theta)\geq v_{1}(x(\hat{\theta})|\theta)+t(\hat{\theta}),~ \forall \theta, \hat{\theta} \notag
\end{align}

\begin{enumerate}
    \item Verify that the first best solution satisfies $d x(\hat{\theta})/d \hat{\theta}\geq 0$.
    \item Verify that the second best solution satisfies $d x(\hat{\theta})/d \hat{\theta}\geq 0$.
    \item In class, we solve the second best solution with $SC_{+}$. What happens if we assume $SC_{-}$ instead?
\end{enumerate}

\end{HW}


\begin{HW}
    In the example of regulating a firm, using integration by parts to rewrite the double integral $$\int_{\underset{\bar{}}{\theta}}^{\bar{\theta}} \int_{\theta}^{\bar{\theta}}\phi'(\tau-c(\tau))d\tau g(\theta)d\theta.$$ 
\end{HW}

\begin{HW}
    \begin{theorem}
        With linear preferences, $f$ is BIC, if and only if 
        \begin{enumerate}
            \item $V_{i}$ is non-decreasing in $\theta_{i}$
            \item $T_{i}(\theta_{i})=\int_{\underset{\bar{}}{\theta_{i}}}^{\theta_{i}}V_{i}(\tau)d\tau-\theta_{i}V_{i}(\theta_{i})+u_{i}(\underset{\bar{}}{\theta_{i}})$
        \end{enumerate}
    \end{theorem}
Prove that the ``if'' part of this theorem by contradiction.
\end{HW}

\begin{HW}
    We've shown that the optimal mechanism is equivalent to first price auction with reserve price. With $\sum_{i}x_{i}\leq 1$, the seller can keep the good. What would be the optimal mechanism with $\sum_{i}x_{i}= 1$?
\end{HW}


\begin{HW}
\begin{enumerate}
    \item  Prove that the revelation principle holds for BNE.
    \item Could there be any bad equilibrium? YES (Read Example 4.1.2 in Dasgupta, Hammond \& Maskin (1979))
\end{enumerate}
\end{HW}


\begin{HW}
\begin{enumerate}
    \item  Prove that the revelation principle holds for dominant strategy equilibrium.
\end{enumerate}
\end{HW}

\begin{HW}

We can write transfers in different ways:

DSIC $\implies$ [FOC] $\frac{\partial t_{i}}{\partial \hat{\theta}_{i}}=-\frac{\partial v_{i}}{\partial y}\frac{\partial y}{\partial \hat{\theta}_{i}}$ $\implies$ $t_{i}(\theta_{i},\hat{\theta}_{-i})=t_{i}(\underset{\bar{}}{\theta_{i}},\hat{\theta}_{-i})-\int _{\underset{\bar{}}{\theta_{i}}}^{\theta_{i}}\frac{\partial v_{i}}{\partial y}\frac{\partial y}{\partial \hat{\theta}_{i}}d\tau$

DSIC $\implies$ [using Envelope Theorem] $t_{i}(\theta_{i},\hat{\theta}_{-i})=u_{i}(\underset{\bar{}}{\theta_{i}},\hat{\theta}_{-i})-v_{i}(y(\theta_{i},\hat{\theta}_{-i})|\theta_{i})+\int _{\underset{\bar{}}{\theta_{i}}}^{\theta_{i}}\frac{d v_{i}}{d \theta_{i}}d\tau$

Are they the same? Why?
\end{HW}

\begin{HW}
If we add marginal cost $K$ and consider budget balance $\sum_{i}t_{i}=-Ky$, how should we modify VCG/Groves mechanisms?
\end{HW}

\begin{HW}
    Is Arrow-d'Aspremont-Gerard-Varet mechanism DSIC?
\end{HW}

\begin{HW}
Show that in the Groves \& Ledyard mechanism, every NE $m^{*}$ gives a P.O. outcome (+BB).
\end{HW}

\end{document} 